\documentclass{bioinfo}
\copyrightyear{2015} \pubyear{2015}

\access{Advance Access Publication Date: Day Month Year}
\appnotes{Manuscript Category}

\usepackage{cite}
\usepackage{hyperref}
\hypersetup{colorlinks=true,linkcolor=blue,urlcolor=blue}

\begin{document}
\firstpage{1}

\subtitle{Subject Section}

\title[short Title]{\textit{koios}: machine learning for big biological data}
\author[Sample \textit{et~al}.]{Simon Dirmeier\,$^{\text{\sfb 1}}$ and Niko Beerenwinkel\,$^{\text{\sfb 1,}*}$}
\address{$^{\text{\sf 1}}$Department, Institution, City, Post Code, Switzerland}

\corresp{$^\ast$To whom correspondence should be addressed.}

\history{Received on XXXXX; revised on XXXXX; accepted on XXXXX}

\editor{Associate Editor: XXXXXXX}

\abstract{\textbf{Motivation:} Analysing biological data is becoming increasingly difficult due to its ever increasing volumnes and dimensionality. Since desktop solutions are not sufficient anymore to analyse big biological data sets comprising terabytes of data, such as features extracted from images or gene expression of millions of cells, we propose a novel machine learning command line tool called \textit{koios} for a distributed highly-parallel analysis of biological data.\\
\textbf{Availability and implementation:} \textit{koios} is available from Github (\url{https://github.com/cbg-ethz/koios}) and soon to be released on {Bioconda}.\\
\textbf{Contact:} \href{niko.beerenwinkel@bsse.ethz.ch}{niko.beerenwinkel@bsse.ethz.ch}}

\maketitle

\section{Introduction}

The not-so-recent advent of high-dimensional data is still posing not only statistical, but primarily methodological problems for researchers across fields such as computational biology, astronomy or the social sciences. \citep{a reference}.
Recently, approaches such as \citep{SCANPY} have been introduced to tackle analysing high dimensional data sets. However, scanpy only scales up to a (few) million observations rendering it unsuitable for analysis of, e.g., microscopy imaging data that often has images of billions of cells. Approaches that also rely on Apache Spark, such as, \citep{biospark} lack ease of use and a capability of effortless data analysos.\\
To summarize, the contributions of this paper are:
\begin{itemize}
	\item A command line tool for analysis of billions of observations.
	\item Clustering and regression using MLLib from Apache Spark.
	\item A proper landing page.
	\item Customizable and automatic builds for single analyses
\end{itemize}



\begin{methods}
\section{Methods}
\textit{koios} is a Python package
callable as executable from the command line that implements various machine learning algorithms for automated data analysis. When using koios the user merely needs to specify input and ouput data and the algorithms to be used for analysis as config file. Running koios will automatically work through the different algorithms. On the backend koios uses Snakemake \citep{snakemake} to automatically build a pipeline of algorithms and methods to call and dynamically composes the correct input and ouput data. Snakemake itself calls Apache's Spark which distributes the data on several workers and runs the respective models.

koios comes with a variety of already implemented algorithms (Figure \ref{fig:workflow}):
\begin{itemize}
\item Dimensionality reduction: PCA, kPCA, factor analysis, 
\item Supervised-learning: regression and random forests,
\item Unsupervised-learning: Gaussian mixture models and $k$-means clustering.
\end{itemize}.

\subsection{Regression example}
Table~\ref{tab:regression}


\subsection{Clustering example}

For instance, when working with single-cell features extracted from microscopy images, one usually wants to first eliminate highly-correlated features, for instance, by embedding into a lower uncorrelated space using PCA, kPCA or factor analysis. Following up clustering

TODO mention not on desktop computer -> needs drivers and so.

\end{methods}

TODO imare of workflow? or image of runtime?
link to website

\section{Conclusion}

\textit{koios} is a commandline tool for machine learning for big biological data sets scaling up to hundreds of millions of cells. koios automatically parses pipelines from a config file and distributes jobs to worker nodes. We believe koios can be a valuable tool since data are possibly getting even larger and hope to establish a community driven effort for working with biological data that cannot be analysed on desktop computers and laptops

\section*{Acknowledgements}

\section*{Funding}

This work has been supported by the ... Text Text  Text Text. \vspace*{-12pt}

\bibliographystyle{natbib}
\bibliography{references}

\end{document}
